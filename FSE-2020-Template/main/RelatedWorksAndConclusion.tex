
\section*{Related Works}

The characteristics of our parallel corpus appeared in other SE problems \cite{027,028}. In these works, SMT outperforms NMT in accuracy. In general, SE researches have specific characteristics of corpus, which brings rooms for deep learning and machine learning to  improve their approaches. Other researches show drawbacks of Machine Learning in SE. \cite{050} and point out the drawbacks of machine learning approach for method name recommendation that is usually suggest too simple method names. \cite{022} shows that the original SMT has problems of exponential time increasing with big data. For the code suggestion area, other research works focuses on a specific types of code tokens. \cite{051} suggested method name based on Hierachical Attention Networks, and \cite{052} suggested method name and class name. In our work, we intend to generate all types of tokens based on writing the abbreviations or prefixes. 


\section*{Conclusion}
In this work, we propose PrefixMap, a code suggestion tool for all types of code tokens in Java programming language. To realize our idea, we propose two Machine Translation models, Statistical Machine Translation and Neural Machine Translation, which learn the information from source language as the space of abbreviation or prefix to the target language as actual code tokens. Our work shows that we got an accuracy from 60\% to 90\% for SMT and from 59\% to 83\% for NMT. In the Machine Learning point of view, we reveal a class of parallel corpus which SMT can learn more information and get better accuracy on NMT in Software Engineering. Two of the characteristics of SE parallel corpus are the unknown tokens problem and being consistent on the length and the order of the source and target corpus. 

There are a few limitations of our work. First, the space of all prefixes are rarely happened in the practice. Secondly, we use a very simple approach to treat the unknown token for NMT. As future work, we will study how abbreviations are written by developers to support more types of abbreviation suggestions instead of suggesting only by prefix, and apply optimization of NMT in other area such as \cite{023} to improve the accuracy. The data is available at \cite{053}.
