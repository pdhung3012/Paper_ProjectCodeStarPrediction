%%
%% This is file `sample-sigconf.tex',
%% generated with the docstrip utility.
%%
%% The original source files were:
%%
%% samples.dtx  (with options: `sigconf')
%% 
%% IMPORTANT NOTICE:
%% 
%% For the copyright see the source file.
%% 
%% Any modified versions of this file must be renamed
%% with new filenames distinct from sample-sigconf.tex.
%% 
%% For distribution of the original source see the terms
%% for copying and modification in the file samples.dtx.
%% 
%% This generated file may be distributed as long as the
%% original source files, as listed above, are part of the
%% same distribution. (The sources need not necessarily be
%% in the same archive or directory.)
%%
%% The first command in your LaTeX source must be the \documentclass command.




\documentclass[sigconf]{acmart}
\acmConference[Iowa State University - COMS 574 Draft Report]{574 Report}{April, 2020}{Ames, IA}
\usepackage{mathtools}
\usepackage{listings}
\usepackage{natbib}
\usepackage{hyperref}
\usepackage{url}
\usepackage{times}
\usepackage{helvet}
\usepackage{courier}
\usepackage{graphicx}
\usepackage{array}% http://ctan.org/pkg/array
\usepackage{subcaption}
\usepackage{cleveref}
\captionsetup[subfigure]{subrefformat=simple,labelformat=simple}
\renewcommand\thesubfigure{(\alph{subfigure})}
\usepackage{multirow}
\usepackage{array}
\usepackage{xcolor}
\usepackage{authblk}

\newcolumntype{L}{>{\centering\arraybackslash}m{7cm}}
\newcolumntype{Q}{>{\centering\arraybackslash}m{5
cm}}
\lstset
{ %Formatting for code in appendix
    language=Java,
    frame=single,
    morekeywords={?},
    basicstyle=\footnotesize,
    numbers=left,
    stepnumber=1,
    showstringspaces=false,
    tabsize=1,
    breaklines=true,
    breakatwhitespace=false,
    moredelim=**[is][\color{red}]{@}{@},
    moredelim=**[is][\color{blue}]{^}{^},
    commentstyle=\itshape\color{green!40!black}\bfseries,
}

%%
%% \BibTeX command to typeset BibTeX logo in the docs
\AtBeginDocument{%
  \providecommand\BibTeX{{%
    \normalfont B\kern-0.5em{\scshape i\kern-0.25em b}\kern-0.8em\TeX}}}


\title[CodeBasedStarPrediction]{Code Based Github Projects Popularity Prediction}
\author[1]{Hung Phan}
\affil[1]{Computer Science Department, Iowa State University}
\begin{document}




%%
%% The abstract is a short summary of the work to be presented in the
%% article.
\begin{abstract}
Popularity is one of important metrics of software projects to be evaluated in large scale software repository. The star count is considered as the representation of popularity of software projects in popular systems like Github. Knowing how many stars the project may get after a duration of time is an interesting feature that the repository can give to the developer. There are works on predicting Github project stars based on different algorithms of Machine Learning (ML) approaches. However, these works relies on features represented how developers commit to their project, which are not be able to see when the project is created. In this work, we propose an alternative scenario, which developers upload a project for first time and the ML model needs to predict the popularity (number of stars) of that project. We relied on features built on the most important part of the project: source code and Abstract Syntax Tree. Our tool achieves \texttt{32\%} as accuracy with star range classification and \texttt{600} as Mean Absolute Error for exact star prediction, which shows challenges along with rooms for improvement in future works. We publish the source code and data at \cite{001} and video presentation at \cite{014}.
\end{abstract}

%%
%% The code below is generated by the tool at http://dl.acm.org/ccs.cfm.
%% Please copy and paste the code instead of the example below.
%%

%%
%% Keywords. The author(s) should pick words that accurately describe
%% the work being presented. Separate the keywords with commas.
\keywords{Machine Learning, Support Vector Machine, Gradient Boosting}



%%
%% This command processes the author and affiliation and title
%% information and builds the first part of the formatted document.
\maketitle

\section{Introduction}
Deep Learning (DL) has been applied in different Software Engineering (SE) researches and problems\cite{001,002,003}. It can contribute to all stages of software development life cycle, from requirements extraction, design, implementation to maintenance \cite{001}. Since DL was vastly applied earlier in Natural Language Processing (NLP), a popular trend of applying DL in SE is to consider the input of SE problems as different types of documentation similar to NLP input. According to \cite{003}, there are 3 main types of documentation: Natural Language (NL), Software Documentation (SD) and Programming Language (PL). Based on the requirement of each tasks, the output of research works that used these types of documentation as input is varied by different types of code tokens. For the research works that used NL as input, they tend to find the element of code environment that satisfied the description in NL, which they applied for code search \cite{004,005} and code synthesis \cite{006}. SD is special type of documentation written in NL but contains information about description of the Application Programming Interfaces (APIs) of different programming languages. There is a work in the literature on representing APIs as vector from SD \cite{007}. For the research works used PL as input, they used deep learning translation between PLs \cite{009} and code suggestion \cite{008}.

The output of DL researches in SE problems can not only be the source code or code tokens, but also could be the information from SDs and NLs.  Different applications are proposed to translate between each types of documentations using Machine Learning (ML). There are works on generating SD as pseudo-code from code using DL and ML \cite{010,011}, or generating documentation from API specifications \cite{012}, or generating commit messages in NL \cite{013}. Neural Machine Translation (NMT), which is a technique relied on advantages of DL, can be assumed as the best translation engine for SE. The ability of NMT relies on the formation on multiple layers of neural network to capture more information for the translation of each elements in the source language \cite{014}. Besides, along with text sequence, NMT can be applicable on a different data structure such as graph or tree, which is suitable for the representation of code \cite{015}. Another advantage of NMT is the performance for inferring the results, which is usually outperform earlier Machine Translation techniques \cite{014}.

Before the era of NMT, Statistical Machine Translation (SMT) \cite{016} was the most popular technique for solving SE problems which relied on MT approaches. With the idea of extending the original Bayes rule \cite{017}, SMT provides the ability of learning the context in Natural Language for translation between popular languages. Since the source code also embed information of NL \cite{018}, SMT is successfully be applicable of SE problems as translation between versions of Python \cite{019} and between different PLs \cite{020}. However, compared to newer trends of translation engines such as NMT, SMT reveals 2 drawbacks. First, it cannot learn information from long sequence of text. An implementation tool of SMT, Phrasal \cite{021} can only process the phrase with maximum length of 7. Secondly, the training and testing time of SMT become worse with large training data and increases exponentially \cite{022}. For these reasons, NMT has replaced SMT in SE problems \cite{010,011}.

Although having many advantages, NMT itself contains some challenges which also appear in researches of different areas along with SE. \cite{024,023} mention about an important problem of NMT compared to SMT as rare words problems in NLP. Current popular NMT engines, OpenNMT \cite{025} or Google NMT \cite{026}, cannot handle large size vocabulary with more than 100000 words. To optimize the problem, researchers considered rare words as Unknown words which their translated results are not counted to the final results. This fact caused NMT performs poorly when rare/unknown words are frequent in the corpus \cite{024}. In SE researches, the problem of Type Inference using MT shows that SMT model provided by \cite{028} has a significant higher accuracy compared to the original NMT approach in \cite{027}. Similarly, for natural language diacritic restoration, \cite{029} shows that SMT outperforms NMT.  \cite{026,027,028} have the same characteristics of parallel corpus, i.e., the length of source and target pairs are equal and the order of the source and target words are consistent with each other. This leads us to an assumption that if the parallel corpus for training MT has these characteristics, it will affect the accuracy of NMT. In summary, NMT tends to have lower accuracy than SMT due to the methodology of NMT that didn't support the rare words translation and the characteristics of parallel corpus. 

In this work, we further investigate the efficiency of NMT vs. SMT in a new research problem that has similar characteristics of parallel corpus as \cite{026,027,028}. Instead of focusing on only limited types of tokens in code environment, our problem provide a solution to help developers get the code of all types of code tokens based on its first letters in the form of abbreviation/prefix of tokens. To implement the solution for this problem, we build two spaces of abbreviations of code tokens as source language and code tokens as target language. Then, we implement two machine translation models, Neural Machine Translation and Statistical Machine Translation to learn the mapping from prefixes to code tokens for code suggestion. We also analyze the affect of unknown tokens along with the accuracy on each types of tokens. By the evaluation, we show that SMT outperforms the original NMT. We called our approach PrefixMapping and analyze the effective translation on three types of documentation: NL, SD and PL. Overall, this paper provides the following contributions:

\begin{enumerate}
\item Proposing the translation based engine for code completion from first letters of tokens.
\item Providing algorithms for extracting parallel corpus of prefixes and tokens in 3 types of documentation used in NLP and SE.
\item Implementing and Optimizing Statistical Machine Translation for PrefixMapping.
\item Implementing and Optimizing Neural Machine Translation for PrefixMapping.
\item Analyzing the accuracy of NMT compared to SMT along with accuracy depending on each types of code tokens.
\end{enumerate}

The structure of the paper is provided as follow. In the next section, we will describe about our research problem. In this section, we will define important concepts we use for abbreviations, algorithms for data collection, and the overview architecture of the system. The core engines of translation, NMT and SMT, along with their optimization for this problem is described in the third section. In the evaluation, we will provide the accuracy and head-to-head comparison between NMT and SMT in 3 types of documentation. We published our data and code at \cite{053}.




\section{Related works}
In our knowledge, automatic predicting popularity of star by machine learning has been explored by other course projects and research projects in \cite{002,003,004}. In \cite{004}, this project targeted the prediction to classifying unpopular software projects and popular software projects in Python. The author formulated the problem as binary classification with projects have less than 100 stars and greater than or equal to 100 stars. \cite{002,003} solved the problem in a different angle, with predicting the exact star of projects and use popular ML metrics like R2 score for prediction. In both \cite{002,003,004}, they use the features collected from Github system, which measures the number of commits, number of forked projects, number of developers etc. Though they are good features, they are assumed to be obtained along with the number of stars in a specific time. In the other words, if you know the current number of forked projects and number of commits in a software project, you are also know the current star of projects released by Github at the same time. This caused a drawback in a scenario that when you first published the project to Github, you cannot have those features to predict the stars after a range of time. In our approach, we rely on the source code itself to provide features for machine learning.


\section{Background}
To make this approach feasible, we need to study about important technique about Abstract Syntax Tree, Term Frequency- Inverse Document Frequency (TF-IDF), algorithms in Machine Learning Classification and Machine Learning Regression.
\subsection{Abstract Syntax Tree}
The Abstract Syntax Tree (AST) is one of important data structure used in Compiler Design \cite{006}. AST is the intermediate reprsentation of source code which contains different types of AST nodes in different levels in a parsed tree. AST can be understand as the way programming language editor looks at the source code in a given software project. In our project, we interact with Java AST. We use Eclipse JDT \cite{007} to extract information about Java source code. This tool provide a solution for parsing using the AST parser which defined the syntax of over 82 types of ASTs. 
\subsection{Term Frequency - Inverse Document Frequency}
The power of current traditional Machine Learning model relies on the architecture of learning from input features in form of decimal number to the output as labels or score. While source code is written by textual information, it is needed to have an approach for converting from textual representation to vector representation. 

Bag of Word (BOW) \cite{007} is considered as the traditional technique for vectorization. The idea of BOW is to build a large vocabulary from set of textual units as sentences, paragraphs or documents. Then, for each textual unit, we build vector for it by representing a word appeared in the unit as 1 and 0 otherwise. Though BOW is straightforward and easy to understand, the main drawback of BOW is that it cannot represent the important of each words inside the set of documents.

TF-IDF \cite{008} is invented to overcome this challenge of BOW. To represent the document as vector without avoiding the important of each words in a document, this approach provide the representation by 2 elements. The first element, term frequency (TF), is used to calculate the frequency of a word occurs in a given document. The Inverse Document Frequency (IDF), in the other side, measure how important of the word in the set of documents is. The total score of TF-IDF for each term \texttt{t} in document \texttt{d} is the product of TF score and IDF score. From this, we can calculate each dimension of each terms to conduct the whole vector for document.
\\
In our research, we use an extension of TF-IDF n-grams with n equals to 4-grams. The different between original TF-IDF and 4-grams TF-IDF is the latter calculate the score for a sequences with up to 4 words in a corpus while the original one only calculate for each single words. We inherit Python class \texttt{TfidfVectorizer} in \cite{009}. There are another vectorization techniques using neural embedding as WordToVec, Glove, DocToVec or CodeToVec \cite{013}. Due to the complexity in implementation and training of these approaches, we focus on TF-IDF embedding for this project and leave other embedding techniques for future works.


\subsection{Machine Learning Classification}
In this part, we briefly describe about the main algorithms we use for classification. We use following algorithms: GaussianNB (GNB), Logistic Regression (LR),Decision Tree (DT),
Random Forest (RF), AdaBoost (AB), Linear Discriminant Analysis (LDA),Quadratic Discriminant Analysis (QDA),
LinearSVC (LSVC), MLPClassifier (MLPC), GradientBoosting (GB). These are popular ML models implemented in Scikit-learn \cite{010}. Theories of each ML models can be found in \cite{011,012}. 


\subsection{Machine Learning Regression}
We use following algorithms for regression: Decision Tree Regressor, Random Forest Regressor, AdaBoost Regressor,
XGB Regressor,
LinearSVR, MLPRegressor, GradientBoosting Regressor. Similar to classification models, theories and Python implementation of these regression models can be found in \cite{010,011,012}.



\section{Materials}
\section{Method}
\section{Result}
For preparing the data for training and testing, we split the data in \cite{013} by 2 sets: training and testing. For the training set, we have 9000 projects. We use 305 projects as the testing data set. We use the same training and testing set used in \cite{013} in their research work. To get the label as star information for each projects, we used Github API to make queries for getting projects' metadata. In other words, we use source code from 2018 to predict the star in April, 2020.  We focus on 3 Research Questions:
\begin{enumerate}
    \item \textbf{Research Question (RQ) 1}: How well of the popularity level classification?
     \item \textbf{RQ 2}: How well of the exact star prediction using regression models?
      \item \textbf{RQ 3}: Can hyper parameter tuning improve the accuracy?
\end{enumerate}



\subsection{RQ 1: Accuracy on Popularity Level Classification}

% Please add the following required packages to your document preamble:
% \usepackage[table,xcdraw]{xcolor}
% If you use beamer only pass "xcolor=table" option, i.e. \documentclass[xcolor=table]{beamer}
\begin{table}[]
\begin{tabular}{|l|l|l|}
\hline
{\color[HTML]{000000} \textbf{No}} & {\color[HTML]{000000} \textbf{ML Model}} & {\color[HTML]{000000} \textbf{Accuracy (\%)}}        \\ \hline
{\color[HTML]{000000} 1}           & {\color[HTML]{000000} GNB}               & {\color[HTML]{000000} 22.70} \\ \hline
{\color[HTML]{000000} 2}           & {\color[HTML]{000000} LR}                & {\color[HTML]{000000} 28.62}                         \\ \hline
{\color[HTML]{000000} 3}           & {\color[HTML]{000000} DT}                & {\color[HTML]{000000} 22.37}                         \\ \hline
{\color[HTML]{000000} 4}           & {\color[HTML]{000000} RF}                & {\color[HTML]{000000} 21.05}                         \\ \hline
{\color[HTML]{000000} 5}           & {\color[HTML]{000000} AB}                & {\color[HTML]{000000} 22.70}                         \\ \hline
{\color[HTML]{000000} 6}           & {\color[HTML]{000000} LDA}               & {\color[HTML]{000000} 29.61}                         \\ \hline
{\color[HTML]{000000} 7}           & {\color[HTML]{000000} QDA}               & {\color[HTML]{000000} 25.99}                         \\ \hline
{\color[HTML]{000000} \textbf{8}}  & {\color[HTML]{000000} \textbf{SVC}}      & {\color[HTML]{000000} \textbf{32.89}}                \\ \hline
{\color[HTML]{000000} 9}           & {\color[HTML]{000000} MLP}               & {\color[HTML]{000000} 30.26}                         \\ \hline
{\color[HTML]{000000} 10}          & {\color[HTML]{000000} GBo}               & {\color[HTML]{000000} 28.62}                         \\ \hline
\end{tabular}
\label{tblRQ1}
\caption{Github Project Popularity Level Classification Result}
\end{table}

The result for popularity level prediction is shown on Table \ref{tblRQ1}. We use the accuracy function in \cite{010} to compare between ML models. We tried with 10 different ML models. In these models, we used default configurations by \cite{010}. The result shows challenges of the prediction based on the source code only. We achieved the highest accuracy on Support Vector Machine, while lowest on Random Forest. The Neural Network achieved good results compared to other ML models. The Gaussian Naive Bayes seems to have the best running time, but it achieved low accuracy. This result shows that the Random Forest might need to have parameter tuning in future works.


\subsection{RQ 2: Accuracy on Popularity Prediction}

% Please add the following required packages to your document preamble:
% \usepackage[table,xcdraw]{xcolor}
% If you use beamer only pass "xcolor=table" option, i.e. \documentclass[xcolor=table]{beamer}
\begin{table}[]
\begin{tabular}{|l|l|l|}
\hline
{\color[HTML]{000000} \textbf{No}} & {\color[HTML]{000000} \textbf{ML Model}} & {\color[HTML]{000000} \textbf{MAE}}    \\ \hline
{\color[HTML]{000000} 1}           & {\color[HTML]{000000} DTR}               & {\color[HTML]{000000} 1111.02}         \\ \hline
{\color[HTML]{000000} 2}           & {\color[HTML]{000000} RFR}               & {\color[HTML]{000000} 878.68}          \\ \hline
{\color[HTML]{000000} 3}           & {\color[HTML]{000000} ABR}               & {\color[HTML]{000000} 4366.48}         \\ \hline
{\color[HTML]{000000} 4}           & {\color[HTML]{000000} XGBR}              & {\color[HTML]{000000} 657.27}          \\ \hline
{\color[HTML]{000000} \textbf{5}}  & {\color[HTML]{000000} \textbf{LSVR}}     & {\color[HTML]{000000} \textbf{604.02}} \\ \hline
{\color[HTML]{000000} 6}           & {\color[HTML]{000000} MLPR}              & {\color[HTML]{000000} 786.77}          \\ \hline
{\color[HTML]{000000} 7}           & {\color[HTML]{000000} GBR}               & {\color[HTML]{000000} 836.53}          \\ \hline
\end{tabular}
\label{tblRQ2}
\caption{Github Project Popularity Prediction Result}

\end{table}

From result showed in Table \ref{tblRQ2}, we can have some observations. We use Mean Absolute Error (MAE), which is a popular metric in ML regression to check the correlation between predicted result and expected result. In our research problem, it is the number of stars for each project. First, the predicted stars seems to be very high compared to the expected results in many projects. One of possible reason is that the training data contains many project with extremely high stars with more than 50000 stars, which can lead to imbalanced data problem. Second, the Linear Support Vector Machine and the XGBoost Regressor tends to have the best MAE although they are also worst in running time. This shows the potential to improve the accuracy by tuning parameters, which we show in the last RQ.

\subsection{RQ 3: Tuning Hyper Parameter Results}
\subsubsection{RQ 3.1: Tuning result on Support Vector Machine Classification}
We observe the result in Table \ref{tblRQ31}. It shows that the best kernel function is also the default kernel function. One possible reason is that the Github projects' stars might be in Gaussian distribution. Other reason may be the other two kernel functions are used better in neural network than SVM. 

% Please add the following required packages to your document preamble:
% \usepackage{multirow}
% \usepackage[table,xcdraw]{xcolor}
% If you use beamer only pass "xcolor=table" option, i.e. \documentclass[xcolor=table]{beamer}
\begin{table}[]
\begin{tabular}{|l|l|l|l|l|}
\hline
{\color[HTML]{000000} \textbf{Metric}} & {\color[HTML]{000000} \textbf{Range}}                 & {\color[HTML]{000000} \textbf{Best Param}} & {\color[HTML]{000000} \textbf{Best Acc}}         & {\color[HTML]{000000} \textbf{Origin Acc}}       \\ \hline
{\color[HTML]{000000} C}               & {\color[HTML]{000000} 1}                              & {\color[HTML]{000000} 1}                   & {\color[HTML]{000000} }                          & {\color[HTML]{000000} }                          \\ \cline{1-3}
{\color[HTML]{000000} kernel}          & {\color[HTML]{000000} {[}'rbf', 'poly', 'sigmoid'{]}} & {\color[HTML]{000000} rbf}                 & \multirow{-2}{*}{{\color[HTML]{000000} 32.89\%}} & \multirow{-2}{*}{{\color[HTML]{000000} 32.89\%}} \\ \hline
\end{tabular}
\label{tblRQ31}
\caption{Result on Support Vector Machine Classification's tuning}
\end{table}

\subsubsection{RQ 3.2: Tuning result on XGBoost Regressor}

% Please add the following required packages to your document preamble:
% \usepackage{multirow}
% \usepackage[table,xcdraw]{xcolor}
% If you use beamer only pass "xcolor=table" option, i.e. \documentclass[xcolor=table]{beamer}
\begin{table}[]
\begin{tabular}{|l|l|l|l|l|}
\hline
{\color[HTML]{000000} \textbf{Metric}}   & {\color[HTML]{000000} \textbf{Range}} & {\color[HTML]{000000} \textbf{Best Param}} & {\color[HTML]{000000} \textbf{Best MAE}}        & {\color[HTML]{000000} \textbf{Default MAE}}     \\ \hline
{\color[HTML]{000000} objective}         & {\color[HTML]{000000} reg:linear}     & {\color[HTML]{000000} reg:linear}          & {\color[HTML]{000000} }                         & {\color[HTML]{000000} }                         \\ \cline{1-3}
{\color[HTML]{000000} colsample\_bytree} & {\color[HTML]{000000} 0.3}            & {\color[HTML]{000000} 0.3}                 & {\color[HTML]{000000} }                         & {\color[HTML]{000000} }                         \\ \cline{1-3}
{\color[HTML]{000000} learning\_rate}    & {\color[HTML]{000000} {[}0.1,1{]}}    & {\color[HTML]{000000} 0.1}                 & {\color[HTML]{000000} }                         & {\color[HTML]{000000} }                         \\ \cline{1-3}
{\color[HTML]{000000} max\_depth}        & {\color[HTML]{000000} {[} 1,3,5{]}}   & {\color[HTML]{000000} 5}                   & {\color[HTML]{000000} }                         & {\color[HTML]{000000} }                         \\ \cline{1-3}
{\color[HTML]{000000} alpha}             & {\color[HTML]{000000} {[}10,20{]}}    & {\color[HTML]{000000} 10}                  & {\color[HTML]{000000} }                         & {\color[HTML]{000000} }                         \\ \cline{1-3}
{\color[HTML]{000000} n\_estimators}     & {\color[HTML]{000000} {[}10,100{]}}   & {\color[HTML]{000000} 10}                  & \multirow{-6}{*}{{\color[HTML]{000000} 648.02}} & \multirow{-6}{*}{{\color[HTML]{000000} 657.27}} \\ \hline
\end{tabular}
\label{tblRQ32}
\caption{Result on XGBoost Regression's Tuning}
\end{table}

The result of tuning XGBoost is shown in \ref{tblRQ32}. From this result, we can see that if we make hyper parameter tuning, the MAE can be better significantly. The algorithm tends to work better with lower learning rate. For the max depth, the higher max depth brings better result, since the XGBoost can learn more information. One of surprised thing we see is that with low number of estimator, we achieve better MAE. Higher number of estimators caused significantly increasing in performance, so we expected the XGBoost will get better accuracy with large size estimator. The result reveals that large number of estimators doesn't mean increasing the accuracy.


\section*{Conclusion}

\bibliographystyle{plainurl}
\bibliography{refs}


\end{document}
