\section{Background}
To make this approach feasible, we need to study about important technique about Abstract Syntax Tree, Term Frequency- Inverse Document Frequency (TF-IDF), algorithms in Machine Learning Classification and Machine Learning Regression.
\subsection{Abstract Syntax Tree}
The Abstract Syntax Tree (AST) is one of important data structure used in Compiler Design \cite{006}. AST is the intermediate reprsentation of source code which contains different types of AST nodes in different levels in a parsed tree. AST can be understand as the way programming language editor looks at the source code in a given software project. In our project, we interact with Java AST. We use Eclipse JDT \cite{007} to extract information about Java source code. This tool provide a solution for parsing using the AST parser which defined the syntax of over 82 types of ASTs. 
\subsection{Term Frequency - Inverse Document Frequency}

\subsection{Machine Learning Classification}
\subsection{Machine Learning Regression}

