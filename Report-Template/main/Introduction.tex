\section{Introduction}
Large scale software project repository is a good location for developer to gain knowledge and experiences in programming. First, a developer can share his/her project with partners in every locations of the world by uploading it to a repository like Github, Gitlab or Bitbucket etc. Every developers in the same project can work on their own tasks and submit their works by commit. The commit management ensures that there is no conflict between developers' code. Second, developers cannot only learn from their own project but also learn from other software projects. They can find references from other project when they feel stuck while doing each tasks. In Github, this system provides the search function which allows developers to search both by code and text. Though the search function on Github doesn't to be as easy to use like standard Q\&A system for coding as StackOverflow, it provides important help for developers.

One of most fundamental criteria for searching in Github is to prioritizing high quality projects as the top search results to developers. How high quality of the projects can be measured by the popularity of the project. There are several kinds of popularity are published in Github: the star count, the number of forks and the the number of commits/ developers. In these kinds, the star count is considered as the official score voted by developers which represents the quality of projects. Projects with high number of stars will be prioritized to be shown to end users. Along with that, many research problems use high stars as the criteria they selected for collect a corpus from Github which contains more than 10000 projects in their researches' evaluation. Thus, the number of stars is crucial for developers to know even when they start published the project on Github.

There are prior research works that automatically predict the star of a given software projects. To implement the solution, they extract public features as metadata for each software projects and applying machine learning regression and classification methods to predict the star of the project.  However, one drawback of these approach is that they inherits the features from Github, which sometimes is private depending on the owner of the project. Besides, when developers know how many commits or how many forks they have for their projects, the star ranking of projects are also available so they don't need the prediction anymore.

In this work, I want to focus on a more useful scenario for developers. In this scenario, developers has a private software project and they start to published it onto Github to share their ideas/ knowledge to other researchers. They want to know how many stars they can get in the next two years. In this case, they cannot have any information about metadata of the projects since they just upload their projects. In this paper, I will solve this challenge. Instead of relying on projects' metadata, I rely to build features based on the core elements of the software: the source code. Then I observe the prediction problems by 2 view points: predict the range of stars and predict the exact stars for the project. I provide data extraction algorithms from corpus of over \texttt{9500} software projects published in 2018 to build Machine Learning (ML) models to predict their stars in April, 2020. 
The process of scoring a software project is provided as follows. First, I select Java files with most information in a source code project. Second, we use a Java source code parser to extract information about Abstract Syntax Tree of files. Third, we use Term Frequency - Inverse Document Frequency (TF-IDF) and Principal Component Analysis (PCA) to map textual content from AST to a vector with features. Forth, we apply multiple Machine Learning (ML) classification and Regression algorithms to predict number of stars as well as the range of stars developers can get after 2 years. Moreover, from this research, I realize some important challenges which we will discuss in the end of this paper. This project has the following contributions:
\begin{enumerate}
    \item Providing mechanism for representing big code by vectorization.
    \item Implementing ML Classification and Regression algorithms to predict stars for Github projects.
    \item Analyzing different parameters to see if they affect the accuracy of star prediction in Support Vector Machine and Gradient Boosting.
    \item Proving challenges of predicting popularity based on code along with potential future research problems. 
\end{enumerate}

The structure of this paper is provided as follow. In the second section, I talk about Related Works to this approach. In third section, background of main techniques used in my approach is described. In next section, materials of this project, including data preparation and prepossessing the data are described. Fifth section shows methodology we used to provide the prediction for Github project. The sixth section proposes research questions we use and the results in evaluation. The final section is our observation on the potentials and challenges for future research. 

